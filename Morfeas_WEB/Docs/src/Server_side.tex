\section{Server Side}
At the server side the Morfeas WEB have some PHP scripts that handle requests from the client. This scripts runs via the PHP engine that executed as module of the Apache WEB server.
The PHP scripts located under the ``Morfeas\_php" directory.

At the following subsections the operation for etch file will be explained.

\subsection{Morfeas\_Web\_if.php}
The PHP script file Morfeas\_Web\_if.php contains the functionality to provide the Logstat files at the client request and to receive ISO Channels configuration objects.
The script can be called as ``./morfeas\_php/Morfeas\_Web\_if.php" and accept ``GET" and ``POST" requests.
\subsubsection{GET request}
In every call with ``GET"" request the Morfeas\_Web\_if.php script expecting an argument ``COMMAND" with value the one of the allowed commands.
At the table~\ref{table:Morfeas_Web_if} will present the allowed commands for the Morfeas\_Web\_if.php script.

\begin{table}[h]
\centering
\begin{tabular}{|l|c|c|}
 \hline
 \textbf{Command} & \textbf{Description} & \textbf{Content type}\\
 \hline
 logstats & Return all logstat combined & JSON \\
 logstats\_names & Return the filenames of the available logstat & JSON \\
 loggers & Return the filenames of the logger (Morfeas component)& JSON \\
 opcua\_config & Return the current ISO Channel configuration & JSON\\
 \hline
\end{tabular}
\caption{GET request values for argument ``COMMAND"}
\label{table:Morfeas_Web_if}
\end{table}

In case that an invalid value sent to the script, the response will be a question mark ``?".
If the calling made without the argument ``COMMAND" then the return will be ``HTTP ERROR 404".

\subsubsection{POST request}
The ``Morfeas\_Web\_if.php" script is also accept a ``POST" request in order to accept new configuration for the ISO Channel of the OPC-UA.
The contents for this kind of request is a JSON object with tag name ``data" and contain an array with the configuration for each ISO Channel.
Every position of the array contains a JSON object with the configuration for each of the ISO channel. The structure for the argument of the POST request
presented at listing~\ref{lst:Morfeas_Web_if}.

\begin{lstlisting}[frame=single,caption=Example for argument of POST request for Morfeas\_Web\_if.php,label=lst:Morfeas_Web_if]
{"data":
  [
   {
      "ISOChannel":"Name of the ISO Channel",
      "IF_type":"MDAQ"|"SDAQ"|"IOBOX"|"MTI",
      "Anchor":"The anchored physical input with the current ISO Channel",
      "Description":"Description of the ISO Channel",
      "Max":"Minimum input value",
      "Min":"maximum input value",
      "Unit":"Unit of measurements (Not needed for SDAQ IF_type)"
   },
    ...
  ]
}
\end{lstlisting}

In every success the ``Morfeas\_Web\_if.php" return a JSON object with a report, otherwise return a string with the error that cause it to fail.

\newpage
\subsection{config.php}
The ``config.php" is a PHP script part of the Morfeas WEB where is responsible for each configuration of the computer and the Morfeas System. It's can called from ``./morfeas\_php/config.php",
and accept ``GET" and ``POST" HTTP requests. At the following subsection the functionality and data the can be exchange via this script will be explained in details.

\subsubsection{GET request}
For accept a ``GET" request the ``config.php" is required to received the request with the argument ``COMMAND" that have a specific value that explain the request that the client want.
At the table~\ref{table:GET_config} derived a list with the allowed values and descriptions for the functionality of each one.

\begin{table}[h]
\centering
\begin{tabular}{|l|c|c|}
 \hline
 \textbf{Command} & \textbf{Description} & \textbf{Content type}\\
 \hline
 getbundle & Return a .mbl file with wholly Morfeas System configuration & File \\
 getISOstandard\_file & Make a force download of current ISOStandard file & File \\
 getCurConfig & Return the current Internet and NTP configuration& JSON \\
 timedatectl & Return the output of timedatectl & HTML \\
 getMorfeasConfig & Return the current Morfeas system configuration & XML \\
 getISOstandard & Return an instance of the current ISOStandard file & XML \\
 getCANifs\_names & Return the available CANBus interfaces & JSON \\
 \hline
\end{tabular}
\caption{GET request values for argument ``COMMAND"}
\label{table:GET_config}
\end{table}

In case that the value of argument ``COMMAND" is not valid or if the argument is missing the script will return ``HTTP ERROR 404".

\subsubsection{POST request}

The ``config.php" script accept also POST HTTP requests that caries out configurations for the computer and the Morfeas System.
The script steering the functionality according to the ``CONTENT\_TYPE" of the HTTP message that the client sent to the server.
The data that contained in the POST request message must be compressed with the internally implemented ``compress" function.
At table~\ref{table:POST_config} will be present all the allowed set values for the ``CONTENT\_TYPE" of a POST request.

\begin{table}[h]
\centering
\begin{tabular}{|l|c|c|}
 \hline
 \textbf{Content type} & \textbf{Description} & \textbf{Contents Data}\\
 \hline
 net\_conf & Request with new INET, CAN and/or NTP configuration & JSON \\
 Morfeas\_config & Request with new configuration for Morfeas system & XML\\
 ISOstandard &  Request with new ISOStandard &  XML\\
 Morfeas\_bundle*** & Request with a restoring bundle &  mbl \\
 reboot & Request for computer reboot & Void \\
 shutdown & Request for computer power off & Void \\
 \hline
\end{tabular}
\caption{Content types for POST request}
\label{table:POST_config}
\end{table}

*** Post request with ``Morfeas\_bundle" as Content type does not required any compression on data.

